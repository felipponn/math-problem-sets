\documentclass{article}
\usepackage[brazil]{babel}
\usepackage[letterpaper,top=2cm,bottom=2cm,left=3cm,right=3cm,marginparwidth=1.75cm]{geometry}

\usepackage{amsmath}
\usepackage{graphicx}
\usepackage[colorlinks=true, allcolors=blue]{hyperref}

\title{Lista Especial de Problemas 5}
\author{Jeferson Almir}
\date{}

% página 24 do PDF

\begin{document}
\maketitle

\begin{enumerate}
    \item Quando um certo inteiro positivo é aumentado em 10\%,
    o resultado é outro inteiro positivo,
    cuja soma dos dígitos é 10\% menor.
    Existe algum inteiro positivo com essa propriedade?
    
    \item Comecemos com um inteiro não-negativo $n$.
    A cada movimento, podemos adicionar 9 ao número atual ou,
    se o número contém um dígito 1,
    podemos deletar este dígito.
    Se surgem zeros à esquerda como resultado,
    eles também são deletados.
    É sempre possível encontrar o número $n+1$ num número finito de passos?
    
    \item É possível que os dez dígitos $0,1,2,3,4,5,6,7,8$ e $9$
    sejam ordenados numa linha de tal forma que,
    quaisquer seis dígitos sejam retirados,
    os quatro dígitos restantes, na ordem em que estão,
    formem um número composto?
    
    \item É possível escolher cinco inteiros positivos tais que suas dez somas em pares terminem todas em dígitos distintos?
    
    \item Um múltiplo positivo de $2020$ tem dígitos distintos,
    e se quaisquer dois de seus dígitos trocam de posição,
    o número resultante não é múltiplo de $2020$.
    Quantos dígitos esse número pode ter?
    
    \item É possível que a soma dos dígitos de um inteiro positivo $n$
    seja $100$ enquanto a soma dos dígitos de $n^3$ seja $100^3$?
    
    \item Determine o menor múltiplo positivo de $2017$
    tal que seus quatro primeiros dígitos sejam $2016$.
    
    \item Existe algum inteiro positivo de $2020$ que contém cada um dos dez dígitos
    a mesma quantidade de vezes?
    
    \item Determine todos os inteiros positivos $n$ que possuem um múltiplo de soma dos dígitos $k$
    para todo $k\geq n$.
\end{enumerate}

\end{document}