\documentclass{article}
\usepackage[brazil]{babel}
\usepackage[letterpaper,top=2cm,bottom=2cm,left=3cm,right=3cm,marginparwidth=1.75cm]{geometry}

\usepackage{amsmath}
\usepackage{graphicx}
\usepackage[colorlinks=true, allcolors=blue]{hyperref}

\title{Lista Especial de Problemas 2}
\author{Jeferson Almir}
\date{}

% 20

\begin{document}
\maketitle

\begin{enumerate}
    \item Vinte e cinco dos números $1, 2, \dots, 50$ são escolhidos. Vinte e cinco dos números $51, 52, \dots, 100$ são escolhidos. Não há nenhum par de números escolhidos que difere por $0$ ou $50$. Determine a soma dos $50$ números escolhidos.

    \item Quando cada um de $100$ números é aumentado em $1$, a soma de seus quadrados permanece a mesma. Cada um dos novos números é aumentado em $1$ mais uma vez. Como a soma dos seus quadrados vai mudar desta vez?

    \item É possível existirem dez inteiros positivos dois a dois distintos tais que sua média dividida pelo seu máximo divisor comum é igual a

    \begin{enumerate}
        \item 6?

        \item 5?
    \end{enumerate}

    \item Basílio calcula a soma de vários inteiros positivos consecutivos que começam do 1. Patrícia calcula a soma de 10 potências positivas de 2 consecutivas, não necessariamente começando do 1. É possível que as duas somas sejam iguais?

    \item Pedro escreve uma lista de todas as somas possíveis de subconjuntos de tamanho 7 de um conjunto de 15 inteiros distintos, enquanto Bia escreve uma lista de todas as somas possíveis de subconjuntos de tamanho 8 desse mesmo conjunto. Se eles organizam seus números numa ordem não-decrescente, é possível que as duas listas fiquem idênticas?

    \item A soma de quaisquer dois de $n$ números distintos é uma potência inteira positiva de 2. Qual é o valor máximo de $n$?

    \item De $\{1,2,3,\dots,100\}$, $k$ inteiros são removidos. Dentre os números restantes, é verdade que sempre existem $k$ inteiros distintos que somem 100 se

    \begin{enumerate}
        \item $k=9$?

        \item $k=8$?
    \end{enumerate}

    \item É possível escolher sete inteiros positivos distintos que somem 100 tais que eles são determinados unicamente pelo quarto maior dentre eles?

    \item Em cada uma de 100 cartas, Ana escreve um inteiro positivo. Esses números não são necessariamente distintos. Em $\binom{100}{2}$ cartas, Ana escreve a soma de cada par possível desses números. Em $\binom{100}{3}$ cartas, ela escreve a soma de cada trio possível. Ela continua o processo até finalmente escrever a soma dos 100 números numa única carta. Ela pode enviar algumas dessas $2^{100}-1$ cartas para Boris, sem que haja um par de cartas enviadas com o mesmo número escrito nelas. Boris sabe das regras pelas quais as cartas foram preparadas. Qual é o número mínimo de cartas que Ana deve enviar a Boris para que ele determine os 100 números originais?
\end{enumerate}

\end{document}