\documentclass{article}
\usepackage[brazil]{babel}
\usepackage[letterpaper,top=2cm,bottom=2cm,left=3cm,right=3cm,marginparwidth=1.75cm]{geometry}

\usepackage{amsmath}
\usepackage{graphicx}
\usepackage[colorlinks=true, allcolors=blue]{hyperref}

\title{Lista Especial de Problemas 6}
\author{Jeferson Almir}
\date{}

% 25

\begin{document}
\maketitle

\begin{enumerate}
    \item Peças brancas e pretas são dispostas num tabuleiro $8\times 8$,
    com no máximo uma peça em cada quadradinho.
    Qual é a quantidade máxima de peças que podem ser postas tais que
    cada linha e cada coluna tenha duas vezes mais peças brancas do que pretas?
    
    \item Há uma peça em cada quadradinho de um tabuleiro $10\times 10$.
    Podemos escolher uma diagonal contendo uma quantidade par de peças
    e remover qualquer peça dela.
    Qual é a quantidade máxima de peças que podem ser removidas do tabuleiro por essa operação?
    
    \item O jogo Campo Minado é jogado num tabuleiro $10\times 10$.
    Cada quadradinho ou contém uma bomba ou está vazio.
    Em cada quadrado vazio é registrado a quantidade de bombas dentre as oito casas vizinhas.
    Então, todas as bombas são removidas,
    e novas bombas são adicionadas em todas as casas que antes estavam vazias.
    Então, números são registrados nas novas casas vazias, como antes.
    É possível que a soma de todos os números no tabuleiro seja agora
    maior que a soma de todos os números na configuração anterior?
    
    \item Seis torres são dispostas num tabuleiro $6\times 6$
    de tal forma que não duas que se ataquem.
    É possível que toda casa vazia seja atacada por duas torres
    
    \begin{enumerate}
    \item que estão distando o mesmo?
    
    \item em distâncias diferentes?
    \end{enumerate}
    
    \item Pedro coloca 500 reis num tabuleiro $100\times 50$
    de tal forma que não há dois que se ataquem.
    Berenice então coloca 500 reis em casas brancas de um tabuleiro $100 \times 100$ de tal forma que não há dois que se ataquem.
    Quem tem mais formas de posicionar seus reis?
    
    \item Para quais $k$ é possível posicionar um número finito de rainhas
    nas casas de um tabuleiro infinito,
    de tal forma que a quantidade de rainhas em cada linha,
    coluna ou diagonal seja ou $0$ ou exatamente $k$?
    
    \item Uma quantidade de formigas está num tabuleiro $10\times 10$,
    cada uma numa casa diferente.
    Cada minuto, cada formiga se move para uma casa adjacente,
    ou para o leste, para o sul, para o oeste, ou para o norte.
    Ela continua nessa mesma direção enquanto for possível,
    mas inverte a direção se atingiu a borda do tabuleiro.
    Em uma hora, nunca duas formigas dividem a mesma casa.
    Qual a quantidade máxima de formigas no tabuleiro?
    
    \item Duas formigas andam sobre os lados dos 49 quadrados
    de um tabuleiro $7\times 7$.
    Cada formiga passa por todos os 64 vértices exatamente uma vez
    e retorna seu ponto inicial.
    Qual é a menor quantidade possível de lados cobertos por ambas as formigas?
    
    \item Num tabuleiro $100\times 100$,
    uma formiga começa do canto inferior esquerdo,
    visita o canto superior esquerdo
    e termina o percurso no canto superior direito.
    Ela navega por casas que compartilham algum lado em comum.
    Os movimentos são alternadamente na vertical ou na horizontal,
    sendo o primeiro na horizontal.
    É necessário que existam duas casas adjacentes tais que
    a formiga foi de uma para a outra pelo menos duas vezes?
\end{enumerate}

\end{document}