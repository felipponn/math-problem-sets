\documentclass{article}
\usepackage[brazil]{babel}
\usepackage[letterpaper,top=2cm,bottom=2cm,left=3cm,right=3cm,marginparwidth=1.75cm]{geometry}

\usepackage{amsmath}
\usepackage{graphicx}
\usepackage[colorlinks=true, allcolors=blue]{hyperref}

\title{Lista Especial de Problemas 8}
\author{Jeferson Almir}
\date{}

% 28

\begin{document}
\maketitle

\begin{enumerate}
    \item Num bilhete de loteria, um número de sete dígitos distintos é escrito.
    Na data do sorteio, um número oficial de sete dígitos distintos é revelado.
    Um bilhete ganha o prêmio se ele coincide ao menos um dígito com o número oficial.
    É possível garantir um bilhete ganhador comprando até seis bilhetes?
    
    \item Pedro compra um bilhete de loteria no qual ele escreve um número de $n$ dígitos,
    nenhum deles sendo 0.
    Na data do sorteio, os administradores da loteria revelarão uma tabela $n\times n$,
    em que cada quadrado contém um dos dígitos de 1 a 9.
    Um bilhete ganha o prêmio se ele \textbf{não} coincide com nenhuma linha ou coluna dessa tabela, em qualquer direção.
    Pedro quer pagar propina para os administradores para que revelem os dígitos em alguns quadrados escolhidos por Pedro,
    de tal forma que Pedro possa garantir ter um bilhete vencedor.
    Qual a quantidade mínima de quadrados que Pedro deve conhecer?
    
    \item Um banco tem um milhão de clientes,
    sendo um deles o Inspetor Bugiganga.
    Cada cliente tem uma única senha que consiste de seis dígitos.
    O Dr. Garra tem uma lista de todos os clientes.
    Ele pode hackear a conta de qualquer cliente,
    escolher quaisquer $n$ dígitos da senha e copiá-los.
    Os $n$ dígitos que ele copia de diferentes clientes
    não precisam sempre estar nas mesmas posições.
    Ele pode hackear a conta de cada cliente, mas apenas uma vez.
    Qual é o menor valor de $n$ que permite ao Dr. Garra
    determinar a senha completa do Inspetor Bugiganga?
    
    \item Um dragão tem $41!$ cabeças e o cavaleiro tem três espadas.
    Sua espada de ouro corta metade da quantidade atual de cabeças do dragão, mais uma cabeça adicional.
    Sua espada de prata corta um terço da quantidade atual de cabeças do dragão, mais duas adicionais.
    Sua espada de bronze corta um quarto da quantidade atual de cabeças do dragão, mais três adicionais.
    O cavaleiro pode usar qualquer uma das três espadas em qualquer ordem.
    Entretanto, se a quantidade atual de cabeças do dragão não múltipla de 2 ou 3,
    as espadas não funcionam e o dragão devora o cavaleiro.
    É possível ao cavaleiro matar o dragão cortando todas as suas cabeças?
    
    \item Um dragão dá 100 moedas a um cavaleiro em cativeiro.
    Metade delas são mágicas,
    mas apenas o dragão sabe quais são.
    A cada dia, o cavaleiro divide as moedas em duas pilhas que não são necessariamente de igual tamanho.
    Se cada pilha contém a mesma quantidade de moedas mágicas,
    ou a mesma quantidade de moedas não-mágicas,
    o cavaleiro é liberto.
    É possível para o cavaleiro garantir sua liberdade em até
    
    \begin{enumerate}
    \item 50 dias?
    
    \item 25 dias?
    \end{enumerate}
    
    \item Depois de uma sessão de apostas,
    cada um de cem piratas calcula a quantia que ganhou ou perdeu.
    O dinheiro só pode ser transferido da seguinte forma:
    ou um pirata paga uma quantia igual a cada um dos outros piratas,
    ou um pirata recebe uma mesma quantia de cada um dos outros piratas.
    Cada pirata tem dinheiro suficiente para fazer qualquer pagamento.
    É sempre possível, depois de vários passos como os descritos,
    que todos os vencedores recebam exatamente o que eles ganharam
    e para todos os perdedores pagar exatamente o que perderam?
    
    \item Há 41 letras num círculo;
    cada letra é A ou B.
    Podemos trocar ABA por B e vice-versa,
    assim como trocar BAB por A e vice-versa.
    É sempre possível, usando essas trocas,
    obter um círculo contendo uma única letra?
    
    \item Moedas numeradas de 1 a 100 são dispostas numa fila.
    Por 1 real, podemos trocar duas moedas adjacentes de lugar,
    mas é grátis trocar a posição de duas moedas que possuem exatamente três moedas entre elas.
    Qual é o custo mínimo para deixar as 100 moedas na ordem inversa?
    
    \item Moedas numeradas de 1 a 100 são dispostas numa fila.
    Por 1 real, podemos trocar a posição de moedas adjacentes,
    mas é de graça trocar a posição de duas moedas que possuam exatamente quatro moedas entre elas.
    Qual é o custo mínimo para deixar as 100 moedas na ordem inversa?
\end{enumerate}

\end{document}