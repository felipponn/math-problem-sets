\documentclass{article}
\usepackage[brazil]{babel}
\usepackage[letterpaper,top=2cm,bottom=2cm,left=3cm,right=3cm,marginparwidth=1.75cm]{geometry}

\usepackage{amsmath}
\usepackage{graphicx}
\usepackage[colorlinks=true, allcolors=blue]{hyperref}

\title{Lista Especial de Problemas 13}
\author{Jeferson Almir}
\date{}

% 35

\begin{document}
\maketitle

\begin{enumerate}
    \item Cada um de dez garotos tem 100 cartas.
    A cada movimento, um deles dá uma carta a cada um dos outros.
    Qual é a quantidade mínima de movimentos para que cada um tenha uma quantidade diferente de cartas?
    
    \item Há 100 pontos no plano.
    Todas as 4950 distâncias dois a dois são registradas.
    
    \begin{enumerate}
    \item Um único registro é apagado.
    É sempre possível recuperá-lo usando os outros registros?
    
    \item Suponha que não haja três pontos colineares,
    e que $k$ registros foram apagados.
    Qual é o valor máximo de $k$ tal que a restauração dos valores apagados é sempre possível?
    \end{enumerate}
    
    \item Cem crianças de alturas distintas ficam em fila.
    A cada minuto, um grupo de 50 crianças consecutivas é escolhido,
    e então são reordenados de forma arbitrária.
    É sempre possível, em seis minutos,
    ordenar as crianças de forma decrescente de altura?
    
    \item Dez crianças de alturas distintas formam uma roda.
    A cada movimento, uma delas vai para uma nova posição na roda
    entre outras duas crianças.
    Qual é a quantidade mínima de movimentos para que sempre possamos
    ordernar as crianças em ordem crescente de altura?
    
    \item Num tabuleiro $8\times 8$, em cada casa está enterrado um tesouro ou uma mensagem.
    Há apenas uma casa com um tesouro.
    Uma mensagem indica o número mínimo de passos necessários para ir
    dessa casa até a casa do tesouro.
    Cada passo vai de uma casa para outra casa com um lado comum.
    Qual é a quantidade mínima de casas que devem ser desenterradas
    para que se descubra o tesouro com certeza?
    
    \item Um inteiro $k$ é dado, com $2\leq k\leq50$.
    Numa roda estão 100 pontos brancos.
    A cada movimento, escolhemos um bloco de $k$ pontos adjacentes
    tais que o primeiro e o último são brancos,
    e pintamos eles de preto.
    Para quais valores de $k$ é possível pintarmos todos os 100 pontos de preto após 50 movimentos?
    
    \item Cem filhotes de urso têm $1,2,\dots,2^{99}$ amoras, respectivamente.
    Uma raposa escolhe dois filhotes e divide suas amoras igualmente entre eles.
    Se sobra uma amora, a raposa a come.
    Qual é a quantidade mínima de amoras que a raposa pode deixar para os filhotes?
    
    \item Merlin chama os $n$ cavaleiros de Camelote para uma conferência.
    A cada dia, ele escolhe cada cavaleiro para uma das $n$ cadeiras da Távola Redonda.
    A partir do segundo dia, quaisquer dois vizinhos
    podem trocar de lugar se eles não tiverem sido vizinhos no primeiro dia.
    Os cavaleiros tentam se sentar em alguma ordem circular que já tenha ocorrido num dia anterior.
    Se eles conseguirem, então a conferência acaba nesse dia.
    Qual é o número máximo de dias para o qual Merlin pode garantir que a conferência vai durar?
    
    \item Um diretório de computador lista todos os pares de cidades conectadas por voos diretos.
    Ana hackeia esse computador e permuta os nomes das cidades.
    Acontece que, qualquer que seja a cidade que seja renomeada Moscou,
    ela pode renomear as outras cidades para que o diretório continue correto.
    Depois, Breno faz o mesmo.
    Entretanto, ele insiste em trocar o nome de Moscou com o de outra cidade.
    É sempre possível para ele renomear as outras cidades
    de forma ao diretório continuar correto?
\end{enumerate}

\end{document}