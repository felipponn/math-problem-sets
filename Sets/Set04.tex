\documentclass{article}
\usepackage[brazil]{babel}
\usepackage[letterpaper,top=2cm,bottom=2cm,left=3cm,right=3cm,marginparwidth=1.75cm]{geometry}

\usepackage{amsmath}
\usepackage{graphicx}
\usepackage[colorlinks=true, allcolors=blue]{hyperref}

\title{Lista Especial de Problemas 4}
\author{Jeferson Almir}
\date{}

% página 22 do PDF

\begin{document}
\maketitle

\begin{enumerate}
    \item Oito elefantes estão enfileirados em ordem crescente de peso. 
    Começando do terceiro,
    o peso de cada elefante é igual ao peso total dos dois elefantes imediatamente à frente.
    Um deles esteve doente, e pode ter perdido peso.
    Os outros estão saudáveis e seus pesos continuam os mesmos.
    É possível, em duas pesagens usando uma balança,
    determinar se houve perda de peso, e se sim,
    determinar qual elefante perdeu peso?
    
    \item Dentre 100 moedas numa fila,
    há 26 falsas que formam um bloco consecutivo.
    As outras 74 são reais, e elas pesam o mesmo.
    Todas as moedas falsas são mais leves que as reais,
    mas não necessariamente pesam o mesmo.
    Qual o número mínimo de pesagens numa balança para garantir que
    vamos achar pelo menos uma moeda falsa?
    
    \item O peso de cada uma dentre 100 moedas é 1 grama, 2 gramas, ou 3 gramas,
    e há pelo menos uma de cada tipo.
    É possível determinar o peso de cada moeda usando no máximo
    101 pesagens numa balança?
    
    \item  Dentre 239 moedas, há duas falsas de mesmo peso,
    e 237 reais de mesmo peso, mas diferente do peso das falsas.
    É possível, em três pesagens numa balança, determinar se as moedas falsas são mais pesadas ou mais leves que as reais?
    Não é necessário identificar quais moedas são falsas.
    
    \item Kate tem três moedas autênticas de mesmo peso e duas falsas cujo peso total é o mesmo de duas moedas reais.
    Entretanto, uma delas é mais pesada que uma moeda real, e a outra é mais leve.
    É possível para Kate identificar a moeda mais pesada assim como a mais leve em três pesagens numa balança?
    Ela deve decidir com antecedência quais moedas pesar sem conhecimento do resultado de qualquer pesagem.
    
    \item Há quatro moedas de pesos 1001, 1002, 1004 e 1005 gramas, respectivamente. É possível determinar o peso de cada moeda usando quatro pesagens numa balança?
    
    \item Há quatro moedas de pesos 1000, 1002, 1004 e 1005 gramas, respectivamente.
    É possível determinar o peso de cada moeda usando uma balança em no máximo quatro pesagens se
    
    \begin{enumerate}
    \item a balança é normal?
    
    \item a balança é defeituosa, e seu prato esquerdo é 1 grama mais leve que o direito?
    \end{enumerate}
    
    \item Cada uma de quatro moedas tem como peso uma quantidade inteira de gramas.
    Pode ser usada uma balança que mostra a diferença entre os pesos dos objetos em seus pratos esquerdo e direito.
    É possível determinar o peso de cada moeda usando essa balança quatro vezes, se ela pode errar por 1 grama para mais ou menos em no máximo uma pesagem?
    
    \item Temos uma balança defeituosa que fica em equilíbrio apenas quando a razão de pesos totais de seu prato esquerdo para prato direito é igual a $3:4$.
    Temos também um cubo que pesa 6 $kg$, uma quantidade suficiente de açúcar e sacas de peso desprezível para conter o açúcar.
    Em cada pesagem, você pode botar o cubo ou qualquer saca de açúcar de peso conhecido na balança, e adicionar açúcar na saca até que o equilíbrio seja atingido.
    É possível obter uma saca de açúcar de peso 1 $kg$?
\end{enumerate}

\end{document}