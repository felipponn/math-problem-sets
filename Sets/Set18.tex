\documentclass{article}
\usepackage[brazil]{babel}
\usepackage[letterpaper,top=2cm,bottom=2cm,left=3cm,right=3cm,marginparwidth=1.75cm]{geometry}

\usepackage{amsmath}
\usepackage{graphicx}
\usepackage[colorlinks=true, allcolors=blue]{hyperref}

\title{Lista Especial de Problemas 18}
\author{Jeferson Almir}
\date{}

% 44

\begin{document}
\maketitle

\begin{enumerate}
    \item Minhocas crescem numa taxa de 1 metro por hora. Quando atingem seu comprimento máximo de 1 metro, elas param de crescer. Uma minhoca totalmente crescida pode ser dissecada em duas novas minhocas de comprimentos arbitrários que somam 1 metro. Começando com uma minhoca totalmente crescida, pode-se obter 10 minhocas totalmente crescidas em menos de uma hora?
    
    \item De uma estação de polícia situada numa estrada retilínea infinita em ambas direções, um ladrão roubou o carro da polícia. Sua velocidade máxima é igual a 90\% da velocidade máxima de uma moto da polícia. Quando o roubo é descoberto algum tempo depois, um policial começa a perseguir o ladrão numa moto. Entretanto, ele não sabe para qual direção da estrada o ladrão foi, nem há quanto tempo ele saiu de lá. O policial pode ter certeza de que vai capturar o fugitivo?
    
    \item A mãe de Olga assa 7 tortas de maçã, 7 tortas de banana e 1 torta de cereja. Elas são postas nessa exata ordem sobre o contorno de um prato redondo antes desse prato ser posto num forno micro-ondas. Todas as tortas são em aparência idênticas, mas Olga sabe apenas de suas posições relativas no prato, devido à rotação do micro-ondas. Ela quer comer a torta de cereja. Ela pode provar três delas, uma por vez, antes de se decidir sobre qual torta ela quer. Ela pode garantir que vai ficar com a torta de cereja?
    
    \item Quarenta Ladrões são ranqueados de 1 a 40, e a Ali Baba é também dado o ranque 1. Eles querem cruzar um rio usando um barco. Ninguém pode ficar no barco sozinho, e não é permitido que duas pessoas cuja diferença de ranques ultrapasse 1 fiquem no barco ao mesmo tempo. A travessia é possível?
    
    \item Num país há 100 cidades. Cada par de cidades é conectado por voos diretos tanto de ida quanto de volta, cada um custando a mesma quantia. Queremos visitar todas as outras 99 cidades e então retornar a nossa cidade-natal, de tal forma que o custo médio por voo nessa viagem não ultrapasse o custo médio de todos os voos.
    
    \begin{enumerate}
    \item Essa tarefa é sempre possível?
    
    \item Essa tarefa é sempre possível se nos esquecermos de uma das outras 99 cidades?
    \end{enumerate}
    
    \item Penha escolhe um ponto interior de um dos quadrados de um tabuleiro $8\times 8$. Belchior desenha um sub-tabuleiro, consistindo de um conjunto de um ou mais quadrados, cujo contorno é uma única linha poligonal fechada que não tem se corta. Penha vai então dizer a Belchior se o ponto escolhido está dentro ou fora deste sub-tabuleiro. Qual é o número mínimo de vezes que Belchior tem que fazer isso para determinar se o ponto escolhido é preto ou branco?
    
    \item Cem portas e cem chaves são numeradas de 1 a 100. Cada porta é aberta por uma única chave cujo número difere do número da porto por no máximo um. É possível fazer o pareamento de chaves com portas em $n$ tentativas quando
    
    \begin{enumerate}
    \item $n=99$?
    
    \item $n=75$?
    
    \item $n=74$?
    \end{enumerate}
    
    \item Um prova consiste em 30 peguntas de verdadeiro ou falso. Vitor não sabe nada sobre o assunto dessa prova. Ele pode fazer a prova diversas vezes, com as exatas mesmas questões, e a cada vez ele é informado de quantas perguntas ele acertou. Ele pode ter certeza de que vai acertar as 30 questões
    
    \begin{enumerate}
    \item em sua trigésima tentativa?
    
    \item em sua vigésima-quinta tentativa?
    \end{enumerate}
    
    \item Na entrada de uma caverna há uma mesa redonda que roda. Em cima dessa mesa, há $n$ barris idênticos, espaçados igualmente sobre a borda da mesa. Dentro de cada barril, há uma tilápia ou com sua cabeça para cima ou com sua cabeça para baixo. A cada turno, Ali Baba escolhe de 1 a $n$ desses barris e os vira de cabeça para baixo. E então a mesa roda. Quando ela para de rodar, é impossível dizer quais barris foram virados. A caverna só abre se as cabeças das tilápias nos $n$ barris estiverem todas para baixo ou todas para cima. Determine todos os valores de $n$ para os quais Ali Baba pode abrir a caverna num número finito de turnos.
\end{enumerate}

\end{document}