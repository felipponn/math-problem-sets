\documentclass{article}
\usepackage[brazil]{babel}
\usepackage[letterpaper,top=2cm,bottom=2cm,left=3cm,right=3cm,marginparwidth=1.75cm]{geometry}

\usepackage{amsmath}
\usepackage{graphicx}
\usepackage[colorlinks=true, allcolors=blue]{hyperref}

\title{Lista Especial de Problemas 15}
\author{Jeferson Almir}
\date{}

% 38

\begin{document}
\maketitle

\begin{enumerate}
    \item Cada diagonal de um quadrilátero convexo o divide em dois triângulos isósceles. As duas diagonais do mesmo quadrilátero o dividem em quatro triângulos isósceles. Esse quadrilátero deve necessariamente ser um quadrado?
    
    \item Dos triângulos determinados por 100 pontos numa reta mais um ponto extra fora da reta, no máximo quantos deles podem ser isósceles?
    
    \item É possível dividir todas as retas do plano em pares de retas perpendiculares de forma que cada reta pertença a exatamente um par?
    
    \item Num continente $6\times 6$, 27 dos quadrados são países desenvolvidos e os outras 9 são subdesenvolvidos. Cada país subdesenvolvido tem relações diplomáticas com um país desenvolvido se e só se compartilha ao menos um vértice com ele. É possível que a quantidade de países desenvolvidos em contato com um país subdesenvolvido seja distinta para cada subdesenvolvido?
    
    \item É possível marcar algumas casas de um tabuleiro $19\times19$ de forma que cada tabuleiro $10\times10$ contenha uma quantidade distinta de casas marcadas?
    
    \item Pedro tem um carimbo $n\times n$, $n>10$, tal que 102 dos quadrados unitários são cobertos de tinta preta. Ele usa seu carimbo 100 vezes num tabuleiro $101\times101$, cada vez deixando uma marca de tinta preta em 102 quadrados unitários do tabuleiro. É possível que o tabuleiro fique todo preto com exceção de um quadrado unitário num canto?
    
    \item Um parque tem o formato de um quadrilátero convexo $ABCD$. Alex, Ben e Cris estão caminhando lá, cada um numa velocidade constante. Alex e Ben começam no ponto $A$ ao mesmo tempo, Alex seguindo por $AB$ e Ben por $AC$. Quando Alex chega em $B$, ele imediatamente continua por $BC$. No momento que Alex chega em $B$, Cris começa em $B$, caminhando por $BD$. Alex e Ben chegam em $C$ ao mesmo tempo, e Alex imediatamente continua por $CD$. Ele e Cris chegam em $D$ ao mesmo tempo. É possível que Ben e Cris se encontrem no ponto de interseção de $AC$ com $BD$?
    
    \item Numa rodovia, um pedestre e uma ciclista seguem na mesma direção, enquanto um carro e uma moto vêm da direção oposta. Todos percorrem em velocidades constantes, não necessariamente iguais. A ciclista alcança o pedestre às 10 horas. Depois de um tempo, a ciclista encontra a mota, e após outro intervalo de tempo igual ao primeiro, ela encontra o carro. Após um terceiro intervalo de tempo, o carro encontra o pedestre, e após outro intervalo de tempo igual ao terceiro, o carro alcança a moto. Se o pedestre encontra o carro às 11 horas, quando ele encontra a moto?
    
    \item Um espaçonave pousa num asteróide, que sabemos ser ou um cubo ou uma esfera. A espaçonave envia um explorador que caminha na superfície do asteróide. O explorador continuamente transmite sua posição atual no espaço para a espaçonave, até chegar num ponto simétrico ao local de pouso em relação ao centro do asteróide. Em outras palavras, a espaçonave pode traçar o caminho percorrido pelo explorador. É possível que esses dados sejam insuficientes para a espaçonave determinar se o asteróide é uma esfera ou um cubo?
\end{enumerate}

\end{document}