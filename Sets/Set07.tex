\documentclass{article}
\usepackage[brazil]{babel}
\usepackage[letterpaper,top=2cm,bottom=2cm,left=3cm,right=3cm,marginparwidth=1.75cm]{geometry}

\usepackage{amsmath}
\usepackage{graphicx}
\usepackage[colorlinks=true, allcolors=blue]{hyperref}

\title{Lista Especial de Problemas 7}
\author{Jeferson Almir}
\date{}

% 26

\begin{document}
\maketitle

\begin{enumerate}
    \item Ana e Beatriz jogam um jogo com $n$ peças.
    Ana joga primeiro, e elas jogam então alternadamente.
    A cada jogada, uma jogadora toma para si uma peça
    ou uma quantidade igual a um divisor primo da quantidade atual de peças.
    A jogadora que tomar a última peça ganha.
    Para quais $n$ Ana tem uma estratégia vencedora?
    
    \item Ana e Beatriz joga um jogo com $n>2$ pilhas,
    inicialmente cada uma consistindo de uma única moeda.
    As jogadoras jogam alternadamente, com Ana sendo a primeira.
    A cada turno, uma jogadora escolhe duas pilhas contendo quantidades de moedas primas entre si,
    e une-as numa única pilha.
    A jogadora que não puder jogar perde o jogo.
    Para cada $n$, determine a jogadora com uma estratégia vencedora.
    
    \item Pedro e Beto jogam um jogo com 11 pilhas de 10 pedras cada.
    Pedro começa jogando, e eles jogam então alternadamente.
    Em seu turno, Pedro deve tomar para si 1, 2 ou 3 pedras de qualquer uma pilha.
    Em seu turno, Beto deve tomar uma pedra de 1, 2 ou 3 pilhas.
    Aquele que tomar a última pedra é o vencedor.
    Quem tem uma estratégia vencedora?
    
    \item Dois jogadores tomam turnos escrevendo símbolos em quadrados vazios de um tabuleiro $1\times n$,
    onde $n$ é um inteiro maior que $1$.
    Arão sempre escreve o símbolo X,
    e Gabi sempre escreve o símbolo O.
    Dois símbolos iguais nunca devem ocupar quadrados adjacentes.
    Um jogador que não puder fazer sua jogado é declarado perdedor.
    Se Arão é o primeiro a jogar,
    quem tem uma estratégia vencedora?
    
    \item Alice e Breno jogam um jogo num tabuleiro $1\times (n+2)$.
    Para começar o jogo,
    Alice posiciona uma moeda em qualquer um dos $n$ quadrados interiores.
    Em sua vez, Breno escolhe um inteiro positivo $k$.
    Alice então deve a moeda para o $k$-ésimo quadrado à direita ou à esquerda da sua posição atual.
    Se a moeda acaba fora do tabuleiro, Alice vence.
    Se a moeda para num dos extremos do tabuleiro, Breno vence.
    Se a moeda para num outro quadrado interior, o jogo continua.
    Para quais valores de $n$ Breno tem uma estratégia vencedora num número finito de turnos?
    
    \item Alice e Breno jogam um jogo na reta real.
    Para começar o jogo, Alice posiciona um grão de areia num número $x$ tal que $0<x<1$.
    A cada turno, Breno escolhe um número positivo $d$.
    Alice então deve mover o grão para $x+d$ ou $x-d$.
    Se o grão para em $0$ ou $1$, Breno vence.
    Senão, o jogo procede para o próximo turno.
    Para quais valores de $x$ Breno tem uma estratégia que o permite vencer num número finito de turnos?
    
    \item Num tabuleiro $8\times 8$, as linhas são numeradas de 1 a 8
    e as colunas são chamadas por letras de A a H.
    Num jogo de 2 jogadores neste tabuleiro,
    o primeiro jogador tem uma torre branca que começa na casa B2,
    e o segundo jogador tem uma torre preta que começa na casa C4.
    Os dois alternam turnos movendo suas torres.
    A cada turno, uma torre vai para uma casa na mesma coluna ou linha da casa inicial.
    Entretanto, essa nova casa não pode estar sob ataque da outra torre,
    e nenhuma torre pode ter parado nela anteriormente.
    O jogador sem uma jogada possível perde o jogo.
    Qual jogador tem uma estratégia vencedora?
    
    \item Num tabuleiro $8\times 8$, há uma torre em cada uma das casas $(A,1)$ e $(C,3)$.
    Alice começa jogando, seguida por Beto, e então jogam alternadamente.
    A cada turno, quem joga move uma das torres um número qualquer de casas para cima ou para a direita.
    A torre não pode ``passar por cima" ou compartilhar casa com a outra torre.
    O jogador que mover qualquer torre para a casa $(H,8)$ vence.
    Quem dentre Alice e Beto tem uma estratégia vencedora?
    
    \item Ana e Belo jogam um jogo onde Ana joga primeiro e a partir daí eles jogam alternadamente.
    Em seu turno, Ana escolhe um inteiro.
    Em seu turno, Belo escreve no quadro branco ou esse número ou a soma de todas os números já escritos.
    É sempre possível para Ana garantir que em algum momento haverá no quadro 100 vezes escrito o número:
    
    \begin{enumerate}
    \item 5?
    
    \item 10?
    \end{enumerate}
\end{enumerate}

\end{document}