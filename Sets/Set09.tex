\documentclass{article}
\usepackage[brazil]{babel}
\usepackage[letterpaper,top=2cm,bottom=2cm,left=3cm,right=3cm,marginparwidth=1.75cm]{geometry}

\usepackage{amsmath}
\usepackage{graphicx}
\usepackage[colorlinks=true, allcolors=blue]{hyperref}

\title{Lista Especial de Problemas 9}
\author{Jeferson Almir}
\date{}

% 30

\begin{document}
\maketitle

\begin{enumerate}
    \item Cada casa de um tabuleiro $29\times 29$ contém um dos inteiros $1,2,3,\dots,29$,
    e cada um desses inteiros aparece 29 vezes.
    A soma dos números acima da diagonal principal é igual
    ao triplo da soma dos números abaixo dessa diagonal.
    Determine o número da casa central desse tabuleiro.
    
    \item Cada uma das casas de um tabuleiro $5\times 7$ contém um número.
    Pedro sabe apenas que a soma das 6 casas de qualquer retângulo $2\times 3$ ou $3\times 2$ é 0.
    Ele pode perguntar pelo número em qualquer posição do tabuleiro.
    Qual é a quantidade mínima de perguntas que ele deve fazer
    para determinar a soma dos 35 números do tabuleiro?
    
    \item Num tabuleiro $4\times 4$,
    a soma dos números em cada linha e em cada coluna é a mesma.
    Nove dos números são apagados,
    deixando apenas os 7 mostrados a seguir.
    É possível unicamente
    
    \begin{enumerate}
    \item um
    
    \item dois
    \end{enumerate}
    
    dos números apagados?
    
    \begin{center}
    \begin{tabular}{|c|c|c|c|}
    \hline
    1 &   &   & 2\\
    \hline
      & 4 & 5 &  \\
    \hline
      & 6 & 7 &  \\
    \hline
    3 &   &   &  \\
    \hline
    \end{tabular}
    \end{center}
    
    \item É possível preencher um tabuleiro $40\times 41$ com inteiros
    de tal forma que o número de cada casa
    é igual à quantidade de casas vizinhas (compartilham aresta)
    com o mesmo número nelas?
    
    \item Num tabuleiro $1000\times 1000$ preenchido de números,
    a soma dos números em cada retângulo de $n$ casas é a mesma.
    Ache todos os valores de $n$ para os quais todos os números do tabuleiro devem ser iguais.
    
    \item Para quais inteiros $n\geq 2$ é possível
    preencher um tabuleiro $n\times n$ com números reais
    de tal forma que todo inteiro de $1$ a $2n(n-1)$ aparece exatamente uma vez
    como soma de números em duas casas adjacentes?
    
    \item \begin{enumerate}
    \item Um tabuleiro $2\times n$ preenchido de números (com $n>2$)
    é tal que as somas das colunas são todas distintas.
    É sempre possível permutar os números no tabuleiro
    para que as somas de colunas continuem diferentes,
    mas as somas de linhas também sejam diferentes?
    
    \item É sempre possível num tabuleiro $10\times 10$?
    \end{enumerate}
    
    \item Há 64 inteiros positivos nas casas de um tabuleiro $8\times8$.
    Sempre que o tabuleiro é coberto com 32 dominós,
    a soma dos dois inteiros cobertos por cada dominó é única.
    É possível que o maior inteiro no tabuleiro não ultrapasse 32?
    
    \item Cem números distintos são escritos nas casas de um tabuleiro $10\times10$.
    A cada movimento, você pode retirar um retângulo formado pelas casas,
    girá-lo $180^{\circ}$, e então colocá-lo de volta.
    É sempre possível rearranjar os números no tabuleiro
    de forma que eles fiquem em ordem crescente em cada linha
    da esquerda para a direita,
    e em cada coluna, de cima para baixo,
    em não mais que 99 movimentos?
\end{enumerate}

\end{document}