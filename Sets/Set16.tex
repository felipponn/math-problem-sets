\documentclass{article}
\usepackage[brazil]{babel}
\usepackage[letterpaper,top=2cm,bottom=2cm,left=3cm,right=3cm,marginparwidth=1.75cm]{geometry}

\usepackage{amsmath}
\usepackage{graphicx}
\usepackage[colorlinks=true, allcolors=blue]{hyperref}

\title{Lista Especial de Problemas 16}
\author{Jeferson Almir}
\date{}

% 40

\begin{document}
\maketitle

\begin{enumerate}
    \item Um mágico coloca as 52 cartas tradicionais do baralho numa fila, e anuncia previamente que o Três de Paus será a única carta restante após 51 passos. Cada passo consiste no público apontar para uma carta qualquer. O mágico pode ou remover esta carta ou remover a carta na posição correspondente contando do extremo oposto da fila. Quais são as posições iniciais possíveis do Três de Paus que garantem  o sucesso deste truque?
    
    \item O público escolhe duas dentre vinte e nove cartas, numeradas de 1 a 29. O assistente do mágico então escolhe duas das vinte e sete cartas restantes, e pede a um membro do público para levá-las ao mágico, que está em outra sala. As duas cartas são apresentadas ao mágico numa ordem arbitrária. Existe alguma forma do assistente e o mágico combinarem previamente um código que permita ao mágico deduzir quais foram as duas cartas escolhidas pelo público?
    
    \item Conforme o assistente observa, o público põe uma moeda em cada uma de duas dentre 12 caixas enfileiradas. O assistente abre uma caixa que não contém uma moeda e se retira. O mágico entra e abre quatro caixas simultaneamente. Existe algum método que garante que ambas as moedas estarão nas quatro caixas abertas?
    
    \item Conforme o assistente observa, o público põe uma moeda em cada uma de duas dentre 13 caixas enfileiradas. O assistente abre uma caixa que não contém uma moeda e se retira. O mágico entra e abre quatro caixas simultaneamente. Existe algum método que garante que ambas as moedas estarão nas quatro caixas abertas?
    
    \item O público permuta $n$ moedas numa fila. A sequência de caras e coroas é escolhida arbitrariamente. O público também escolhe um número de $1$ a $n$, podendo ser $1$ ou $n$. Então, o assistente vira uma das moedas, e o mágico é trazido para examinar a sequência resultante. Por um combinado feito anteriormente, o mágico tenta determinar o número escolhido pelo público.
    
    \begin{enumerate}
    \item Se isso é possível para algum $n$, é também possível para $2n$?
    
    \item Determine todos os $n$ para os quais isto é possível.
    \end{enumerate}
    
    \item Um hotem tem $n$ quartos desocupados em seus andares superiores, $k$ deles em reforma. Todas as portas estão fechadas, e é impossível dizer se um quarto está ocupado ou em reforma sem abrir sua porta. Há 100 turistas na área de recepção do térreo. Cada um deles sobre para abrir a porta de algum quarto. Se está em reforma, a pessoa então fecha essa porta e abre a porta de outro quarto, repetindo o processo até abrir a porta de um quarto que não está em reforma. O turista então se muda para esse quarto e fecha a porta. Cada turista escolhe as portas a serem abertas. Para cada $k$, determine o menor $n$ para o qual os turistas podem concordar numa estratégia enquanto estão na recepção que permita que não haja dois deles no mesmo quarto.
    
    \item Onze magos estão sentados em roda. Um inteiro positivo distinto que não excede 1000 é colado na testa de cada um. Um mago pode ver os números dos outros 10, mas não o seu. Simultaneamente, cada mago ergue a sua mão esquerda ou direita. Então cada um declara o número em sua testa ao mesmo tempo. Há alguma estratégia a qual os magos podem planejar previamente de forma a permitir cada um deles a adivinhar corretamente seu número?
    
    \item Mil magos estão em pé em fila. Cada um veste um dos chapéus numerados de 1 a 1001 em alguma ordem, com um chapéu sobrando. Cada mago pode ver os números dos chapéus à sua frente, mas não pode ver os que estão atrás. Começando do fim da fila, cada mago exclama um número de 1 a 1001 de forma que todos os outros escutam. Cada número pode ser chamado no máximo uma vez. No fim, um mago que falhar em gritar o número de seu chapéu é expulso do Conselho de Magos. Este processo é conhecido pelos magos, e eles têm uma chance de discutir uma estratégia. Há alguma estratégia que consiga manter no Conselho de Magos
    
    \begin{enumerate}
    \item mais de 500 desses magos?
    
    \item ao menos 999 desses magos?
    \end{enumerate}
    
    \item Cada um de $n$ magos numa fila veste um chapéu branco ou preto escolhido com probabilidade uniforme. Cada um consegue ver os chapéus à sua frente, mas não o próprio. Começando do último mago, cada um adivinha a cor de seu chapéu. Além disso, cada mago, exceto o primeiro, anuncia um inteiro positivo que é escutado por todos. Os magos podem decidir previamente por uma estratégia coletiva sobre o número a ser anunciado por cada um, de forma a maximizar o número de palpites corretos. Infelizmente, alguns dos magos não se importam com isso, e podem fazer o que bem entenderem. Não se sabe quem eles são, mas há exatamente $k$ deles. Qual é o número máximo de palpites corretos que podem ser garantidos por uma estratégia coletiva, apesar das possíveis ações dos magos apáticos?
\end{enumerate}

\end{document}