\documentclass{article}
\usepackage[brazil]{babel}
\usepackage[letterpaper,top=2cm,bottom=2cm,left=3cm,right=3cm,marginparwidth=1.75cm]{geometry}

\usepackage{amsmath}
\usepackage{graphicx}
\usepackage[colorlinks=true, allcolors=blue]{hyperref}

\title{Lista Especial de Problemas 10}
\author{Jeferson Almir}
\date{}

% 31

\begin{document}
\maketitle

\begin{enumerate}
    \item Trinta e nove números não-nulos são escritos em fila.
    A soma de quaisquer dois números adjacentes é positiva,
    enquanto a soma de todos os números é negativa.
    O produto de todos os números é negativo ou positivo?
    
    \item Qual é a quantidade máxima de inteiros distintos que podemos dispor numa fileira
    tal que a soma de quaisquer onze inteiros adjacentes é ou 100 ou 101?
    
    \item Vários inteiros positivos são dispostos em fileira.
    Sua soma é 2019.
    Nenhum deles é igual a 40,
    e a soma de qualquer bloco de números adjacentes nunca é igual a 40.
    Qual é o comprimento máximo da fileira?
    
    \item Os inteiros positivos são dispostos numa fileira em alguma ordem,
    cada um aparecendo exatamente uma vez.
    Sempre existe algum bloco de ao menos dois números adjacentes,
    em algum lugar da fileira,
    de tal forma que a soma dos números desse bloco é um número primo?
    
    \item Ao redor de um círculo estão os inteiros de 1 a 33,
    em alguma ordem.
    A soma de cada par de números adjacentes é computada.
    É possível que essas somas sejam 33 inteiros consecutivos?
    
    \item Ao redor de um círculo estão 2015 inteiros positivos
    tais que a diferença entre quaisquer dois adjcentes é igual
    ao seu máximo divisor comum.
    Determine o valor maximal de um número positivo que divide o produto desses 2015 números.
    
    \item Ao redor de um círculo estão 999 números,
    cada um 1 ou -1,
    e há ao menos um de cada.
    O produto de cada bloco de 10 números adjacentes é computado.
    Seja $S$ a soma desses 999 produtos.
    
    \begin{enumerate}
    \item Qual é o valor mínimo de $S$?
    
    \item Qual é o valor máximo de $S$?
    \end{enumerate}
    
    \item Num círculo estão 1000 números não-nulos pintados alternadamente de preto e branco.
    Cada número preto é asoma de seus dois vizinhos
    enquanto cada número branco é o produto de seus dois vizinhos.
    Quais são os valores possíveis da soma desses 1000 números?
    
    \item É possível dispor os números $1,2,\dots,100$ num círculo em alguma ordem
    de forma que o valor absoluto da diferença entre quaisquer dois números adjacentes é ao menos 30 e no máximo 50?
\end{enumerate}

\end{document}