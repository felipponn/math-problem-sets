\documentclass{article}
\usepackage[brazil]{babel}
\usepackage[letterpaper,top=2cm,bottom=2cm,left=3cm,right=3cm,marginparwidth=1.75cm]{geometry}

\usepackage{amsmath}
\usepackage{graphicx}
\usepackage[colorlinks=true, allcolors=blue]{hyperref}

\title{Lista Especial de Problemas 3}
\author{Jeferson Almir}
\date{}

% página 21 do PDF

\begin{document}
\maketitle

\begin{enumerate}
    \item É possível escolher 2016 inteiros
    cuja soma e produto são iguais a 2016?
    
    \item Cinco números não-nulos somados por par.
    Cinco das somas são positivas
    e as outras cinco são negativas.
    Se eles são multiplicados em par,
    determine a quantidade de produtos positivos e negativos.
    
    \item Determine todos os inteiros positivos $n$
    para os quais os números $1,2,\dots,2n$
    podem ser postos em pares
    de tal forma de que se calcularmos a soma de cada par,
    o produto dessas somas é um quadrado perfeito.
    
    \item Para quais inteiros positivos $n$
    é possível escolher $n$ inteiros positivos consecutivos
    cujo produto é igual à soma de $n$ outros inteiros positivos consecutivos?
    
    \item É possível dividir todos os inteiros positivos de $100!$,
    incluindo $1$ e $100!$,
    em dois grupos de mesmo tamanho
    tais que o produto dos números de cada grupo é o mesmo?
    
    \item Existe algum inteiro positivo $n$
    tal que para quaisquer reais $x$ e $y$,
    existem $n$ números reais
    tais que $x$ é igual a sua soma
    e $y$ é igual à soma de seus inversos?
    
    \item Ana escreve vários 1s,
    pondo sempre um sinal de $+$
    ou um sinal de $\times$
    entre cada par de 1s adjacentes,
    e ponde vários pares de parênteses na expressão.
    Sabemos que o valor final dessa expressão é 2014.
    Boris então troca todos os sinais de $+$ por sinais de $\times$ e vice-versa.
    É possível que seua expresão também seja igual a 2014?
    
    \item Gregório escreve 100 números num quadro negro
    e calcula seu produto.
    Em cada movimento,
    ele aumenta cada número em 1
    e calcula seu produto.
    Qual a quantidade máxima de movimentos
    que Gregório pode realizar
    de forma que o produto nunca mude?
    
    \item As 2016 somas de pares dentre 64 números
    são escritas num pedaço de papel.
    Elas são todas distintas e positivas.
    Os 2016 produtos de pares dentre os mesmos 64 números
    são escritos num outro pedaço de papel.
    Elas também são todas positivas e distintas.
    Depois, se esquece qual pedaço de papel é qual.
    É possível determinar qual é qual?
\end{enumerate}

\end{document}