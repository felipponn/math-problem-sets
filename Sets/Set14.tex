\documentclass{article}
\usepackage[brazil]{babel}
\usepackage[letterpaper,top=2cm,bottom=2cm,left=3cm,right=3cm,marginparwidth=1.75cm]{geometry}

\usepackage{amsmath}
\usepackage{graphicx}
\usepackage[colorlinks=true, allcolors=blue]{hyperref}

\title{Lista Especial de Problemas 14}
\author{Jeferson Almir}
\date{}

% 37

\begin{document}
\maketitle

\begin{enumerate}
    \item O número 4 tem uma quantidade ímpar de divisores positivos ímpare, apenas o 1, e uma quantidade par de divisores positivos pares, o 2 e o 4. Existe algum número com uma quantidade ímpar de divisores positivos pares e uma quantidade par de divisores positivos ímpares?
    
    \item Cada um de três inteiros positivos é um múltiplo do máximo divisor comum dos outros dois, e um divisor do mínimo múltiplo comum dos outros dois. Esses três números devem ser iguais?
    
    \item Seja $p$ um número primo. Um conjunto de $p+2$ inteiros positivos, não necessariamente distintos, é chamado de \textit{interessante} se a soma de quaisquer $p$ deles é divisível por cada um dos outros dois. Determine todos os conjuntos interessantes.
    
    \item \begin{enumerate}
    \item Pedro e Bruno pensam, cada um, em três inteiros positivos. Para cada par de seus números, Pedro escreve o MDC dos dois números. Para cada par de seus números, Bruno escreve o MMC dos dois números. Se ambos Pedro e Bruno escreveram os mesmos três números, esses três números escritos devem ser iguais entre si?
    
    \item O resultado análogo é verdade se Pedro e Bruno pensam, cada um, em quatro inteiros ppositivos, no lugar de três?
    \end{enumerate}
    
    \item Seja $a\land b$ o número $a^b$. A ordem das operações na expressão $7\land 7\land 7\land 7\land 7\land 7\land 7$ deve ser determinada inserindo cinco pares de parênteses. É possível pôr os parênteses de duas formas distintas de forma que as expressões tenham valores iguais?
    
    \item Parênteses devem ser inseridos na expressão $10\div9\div8\div7\div6\div5\div4\div3\div2$ de forma que o número resultante seja inteiro.
    
    \begin{enumerate}
    \item Determine o valor máximo desse inteiro.
    
    \item Determine o valor mínimo desse inteiro.
    \end{enumerate}
    
    \item Temos uma cópia de 0 e duas cópias de cada um de $1,2,\dots,n$. Para quais $n$ podemos ordenar os $2n+1$ números, de tal forma que para $1\leq k\leq n$, há exatamente $k$ outros números entre as cópias do número $m$?
    
    \item O Barão de Munchausen tem um conjunto de 50 moedas. A massa de cada uma é um inteiro positivo distinto que não excede 100, e a massa total é par. O Barão diz que não é possível dividir as moedas em duas pilhas de mesma massa total. É possível que o Barão esteja certo?
    
    \item Um conjunto de pelo menos dois objetos com pesos distintos tem a propriedade de que para qualquer par de objetos do conjunto, podemos escolher um subconjunto dos objetos restantes de tal forma que o peso total é igual ao peso total do par. Qual é o número mínimo de objetos neste conjunto?
\end{enumerate}

\end{document}